\documentclass[a4paper]{bxjsarticle}  
\usepackage{zxjatype}
\usepackage[ipa]{zxjafont}

%\documentclass[a4paper]{article}

\usepackage[english]{babel}
%\usepackage[utf8]{inputenc}
\usepackage{amsmath}
\usepackage{graphicx}
\usepackage[colorinlistoftodos]{todonotes}

% \todo[inline, color=green!40]{sample for inline TODO's}
% \todo{sample for margin TODO's}

\title{The Empty Flow Minimization Problem for a Package Delivery Company}

\author{Hideki Hashimoto and Jo\~{a}o Pedro Pedroso and Mikio Kubo}

\date{February 2016	}

\begin{document}
\maketitle

\begin{abstract}
Your abstract.
\end{abstract}

\section{Introduction}

\todo[inline, color=green!40]{Full-Truck Load Vehicle Routing Problem}

問題は感じだね.

\section{Problem Definition}

A large package delivery company in Japan has to design a network composed of the set of centers and the set of bases.
All the packages from customers are delivered to a center, moved to a base, ...


$C$ : set of centers

$B$ : set of bases 

$V$ : set of vehicles

$V_i$: set of vehicles whose home base is $i$

$D_{ij}$ : demand  from base $i$ to base $j$ 

$Q_v$: capacity of vehicle $v$ 

$E_{ij}$ : cost of empty from from $i$ to $j$ 


\section{Approximate Algorithm}


In this section, we propose a quick approximate algorithm for the empty flow minimization problem. 

In the network design problem in which we determine the assignment of each center to a base,
we need to solve the empty flow minimization problem repeatedly. 
So the algorithm proposed here is used to get a quick (approximate) solution. 
\[\]
If we assume  that the vehicles are homogeneous and have the same capacity $Q$, 
the number of vehicles required from base $i$ to $j$ is computed in the following way:
\[
 R_{ij} = \lceil D_{ij}/Q \rceil . 
\]

Then we compute the surplus  (negative number means deficit) of vehicles for base $i$ as follows:
\[
 s_i =  \sum_{j} R_{ji}  -\sum_{j} R_{ij} 
\]


If we use an integer variable $X_{ij}$ that are moved from $i$ to $j$, 
the problem is reduced to the following problem. 
\[
 \min \sum  E_{ij} X_{ij} 
\]
subject to 
\[
  \sum_{j}  y_{ij} = s_i \ \ \ \mbox{ for  all } i \in B
\]

The problem is a transportation problem that can be solved easily (without using integer variables). 


\section{Column Generation Approach}

Go-when-full policy:

Since demands are uncertain and dynamic, the company applies the go-when-full policy 
in which the vehicles waiting at the base go to their destination as soon as their capacity become full. 

Priority assumptions:

1. Owned vehicles have a priority to subcontracted vehicles.

2. For the same types of vehicles, the larger vehicles have a priority to smaller vehicles. 

There are two types of vehicles; one is the owned vehicles, and another is the subcontracted vehicles. 
The owned vehicles have no fixed cost, the subcontracted vehicles require a fixed cost.


Under the assumptions above, we can get the delivery schedule between two bases. 

$M$ : set of all the movements of the vehicles (defined as the departure time of the origin base and the arrival time of the destination base)

${\cal R}_{iv}$ : set of routes for vehicle $v$ whose home base is $i$.

$a_{rm}$ : $=1$ if route $r$ covers movement movement $m$

\[
 \min \ \ \sum_{r} R_r x_r
\]
subject to 
\[
  \sum_{r \in {\cal R}_{iv} } x_r =1 \ \ \  \mbox{ for all } v,i
\]

\[
 \sum_{r} a_{rm} x_r \leq 1 \ \ \  \mbox{ for all } m \in M
\]





\end{document}